%!TEX root = ../main.tex
\documentclass[main]{subfiles}

\begin{document}

\chapter{調査概要}

\section{目的}

本研究の目的は形式手法を適用したソフトウェア開発に特有の問題を調査することである.
そのためにソフトウェアに対して報告される問題に着目し,問題をトピックモデルにより分類することで形式手法の適用によるソフトウェア開発に特有の問題を調査した.
% トピックモデルにより?BERTopicにより?

\todo{RQを書く}

\section{対象}

本研究ではGitHubで公開されているリポジトリのIssueをソフトウェアに対して報告される問題として調査した.
調査対象とするリポジトリとして形式手法を適用して開発されたものとそうでないものとの双方についてそれぞれ複数を選定し,選定したリポジトリからIssueを取得した.
Issueの取得にあたっては次の条件を満たすリポジトリを選定した.
ただし形式手法を適用して開発されたものについては,ドキュメントに形式手法を適用した旨が記述されていることを条件として加えている.

% textlint-disable ja-technical-writing/no-doubled-joshi
\begin{enumerate}
	\item Starの数が100以上である
	\item Issueの数が100以上である
	\item ドキュメントが英語で記述されている
\end{enumerate}

これにより選定したリポジトリの一覧を表\ref{tab:repository_formal}, \ref{tab:repository_common}に示す.

\subfile{repository}

\end{document}

% トピックモデリングの結果得られたトピックを手でラベリングして
% 要求・設計・コーディング・テスト・保守の5種類に分類してみて,その分布の違いを調べようかなと思った
% どんな手法があったっけ?
% カイ2乗検定とか使ってみたい
