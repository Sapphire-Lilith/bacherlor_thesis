%!TEX root = ../main.tex
\documentclass[main]{subfiles}

\begin{document}

\chapter{調査手法}

形式手法を適用したソフトウェア開発に特有の問題を調査するため,\ref{sec:survey-target}節で述べたリポジトリのIssueを各々について分類し,その傾向を比較した.
調査は以下の手順で行った.

% textlint-disable ja-technical-writing/sentence-length
\begin{enumerate}[label=手順\arabic*.]
	\item \label{enum:preprocess}
	      各リポジトリのIssueを取得し,前処理を行う.
	\item \label{enum:bertopic}
	      前処理が施されたIssueをBERTopicにより分類する.
	\item \label{enum:labeling}
	      \ref{enum:bertopic}で得られた分類に対してラベルを付与する.
	\item \label{enum:aggregate}
	      \ref{enum:labeling}で得られたラベルの出現頻度を,形式手法を適用して開発されたソフトウェアとそうでないソフトウェアとのそれぞれについて集計する.
	\item \label{enum:statistical}
	      \ref{enum:aggregate}で得られた集計結果をピアソンの\(\chi^2\)検定およびウェルチの検定により比較する.
\end{enumerate}
% textlint-enable ja-technical-writing/sentence-length

% ========== ========== ========== ========== ========== %
%                       separator                        %
% ========== ========== ========== ========== ========== %

\section{仮説}
\label{sec:hypothesis}

本研究では各リポジトリのIssueを\ref{sec:labeling}節で述べるerror,bug,requirement,improvement,question,maintenance,otherの8種類のラベルによって分類した.
ここで形式手法を適用して開発されたソフトウェアに特有の傾向があることを仮説とし,加えて次の仮説を立てた.

% textlint-disable ja-technical-writing/sentence-length
\begin{enumerate}[label=仮説\arabic*.]
	\item \label{hyp:1} 形式手法を適用して開発されたソフトウェアはそうでないソフトウェアよりerrorの出現割合が低い.
	\item \label{hyp:2} 形式手法を適用して開発されたソフトウェアはそうでないソフトウェアよりbugの出現割合が高い.
	\item \label{hyp:3} 形式手法を適用して開発されたソフトウェアはそうでないソフトウェアよりimprovementの出現割合が低い.
	\item \label{hyp:4} 形式手法を適用して開発されたソフトウェアはそうでないソフトウェアよりmaintenanceの出現割合が高い.
\end{enumerate}
% textlint-enable ja-technical-writing/sentence-length

\ref{hyp:1}はerrorの出現割合についての仮説である.
errorの出現割合が低いことは,形式手法を適用して開発されたソフトウェアに動作不良の報告が少ないことを示唆する.
したがって\ref{hyp:1}は形式手法を適用して開発されたソフトウェアに動作不良が少ないという仮説である.
これは形式手法を適用して開発されたソフトウェアは仕様が形式的に記述・検証されることで,動作不良が少ないと考えられるために立てたものである.

\ref{hyp:2}はbugの出現割合についての仮説である.
bugの出現割合が高いことは,形式手法を適用して開発されたソフトウェアにバグの報告が多いことを示唆する.
したがって\ref{hyp:2}は形式手法を適用して開発されたソフトウェアにバグが多いという仮説である.
これは形式手法を適用して開発されたソフトウェアは仕様が形式的に記述・検証されることで,バグが発見されやすいと考えられるために立てたものである.

\ref{hyp:3}はimprovementの出現割合についての仮説である.
improvementの出現割合が低いことは,形式手法を適用して開発されたソフトウェアに機能の追加や改善の求める報告が少ないことを示唆する.
したがって\ref{hyp:3}は形式手法を適用して開発されたソフトウェアに機能の追加や改善の要望が少ないという仮説である.
これは形式手法を適用して開発されたソフトウェアは仕様が形式的に記述されることで,仕様の変更が煩雑であると考えられるために立てたものである.

\ref{hyp:4}はmaintenanceの出現割合についての仮説である.
maintenanceの出現割合が高いことは,形式手法を適用して開発されたソフトウェアに保守に関する報告が多いことを示唆する.
したがって\ref{hyp:4}は形式手法を適用して開発されたソフトウェアに保守に関する報告が多いという仮説である.
これは形式手法を適用して開発されたソフトウェアは仕様が形式的に検証されることで,保守が煩雑であると考えられるために立てたものである.

% ========== ========== ========== ========== ========== %
%                       separator                        %
% ========== ========== ========== ========== ========== %

\section{\ref{enum:preprocess}}
\label{sec:preprocess}

本手順ではリポジトリから取得したIssueをBERTopicにより分類するための前処理を行う.
前処理の内容は次の通りである.

% textlint-disable ja-technical-writing/no-doubled-conjunction
\begin{enumerate}
	\item Issueのtitle要素およびbody要素からURL,メールアドレス,コードブロック,記号を除去する.
	\item Issueのtitle要素およびbody要素を結合し,単一のテキストとする.
\end{enumerate}
% textlint-enable ja-technical-writing/no-doubled-conjunction

BERTopicは学習済みの埋め込みモデルを用いるため,一般的な英文の分類においては前処理を必要としない.
しかし本研究で取得したIssueはMarkdown\footnote{https://daringfireball.net/projects/markdown}形式で記述されており,またソフトウェアに対する報告であるため,エラーメッセージやコードスニペットを含むことがある.
これらの情報は文章の意味に影響を与えないが,BERTopicによる分類に影響を与える可能性があるため,前処理を行うことでこれらの情報を除去した.

% ========== ========== ========== ========== ========== %
%                       separator                        %
% ========== ========== ========== ========== ========== %

\section{\ref{enum:bertopic}}
\label{sec:bertopic}

本手順では各リポジトリについて前処理を行ったIssueをBERTopicにより分類する.
この結果として,分類およびその分類に属するIssue数,また各分類を代表する単語および文書が得られる.
ここで分類を代表する単語とは,各分類に含まれる文書の中で最もその分類を特徴づけると考えられる単語であり,本手順においてはそれぞれの分類に対して5単語を抽出した.
また分類を代表する文書とは,各分類に含まれる文書の中で最もその分類を特徴づけると考えられる文書であり,本手順においてはそれぞれの分類に対して3文書を抽出した.

% ========== ========== ========== ========== ========== %
%                       separator                        %
% ========== ========== ========== ========== ========== %

\section{\ref{enum:labeling}}
\label{sec:labeling}

本手順では\ref{enum:bertopic}で得られた分類に対してラベルを付与する.
付与するラベルはHallらの研究\cite{hall:2001}およびBissyand{\'e}らの研究\cite{bissyande:2013}において用いられたものを参考にし,本研究で取得したIssueを分類するために適切であると考えられるものを設定した.
付与したラベルとその意味を表\ref{tab:label_meaning}に示す.

% textlint-disable

\begin{table}[p] % require package "float"
	\centering
	\caption{Issueの分類に付与したラベルとその意味}
	\label{tab:label_meaning}
	\begin{tabular}{cc} % l:left, c:center, r:right
		% "\\" to newlilne, "\hline" to border
		\Hline
		ラベル & 意味 \\\hline
		error & 動作不良を報告するもの \\
		bug & 動作不良の報告であって,明確にバグであると判断されるもの \\
		requirement & ソフトウェアに関して要求するもの \\
		improvement & 要求であって,機能の追加や改善を求めるもの \\
		question & ソフトウェアに関して質問するもの \\
		maintenance & 保守に関するもの \\
		other & 上記のいずれにも分類されないもの \\\Hline
	\end{tabular}
\end{table}

% \end{description}


各ラベルは\ref{enum:bertopic}で得られた分類において,その分類を代表する文書を参考にして付与した.
ただしBERTopicによる分類では外れ値が発生するため,外れ値に対しては一律にotherを付与した.

% ========== ========== ========== ========== ========== %
%                       separator                        %
% ========== ========== ========== ========== ========== %

\section{\ref{enum:aggregate}}
\label{sec:aggregate}

本手順では\ref{enum:labeling}で得られたラベルの出現回数を,形式手法を適用して開発されたソフトウェアとそうでないソフトウェアとに分けて集計する.
まず各リポジトリについて,ラベル付けされた分類に属するIssue数から各ラベルに属するIssueの出現回数を算出し,これを各リポジトリにおけるラベルの出現回数とした.
次に形式手法を適用したソフトウェアとそうでないソフトウェアとのそれぞれのデータセットについて,各リポジトリにおけるラベルの出現回数を合算し,これから各ラベルの出現割合を算出した.
本研究では各データセットについて算出したラベルの出現回数からその出現割合を算出し,これをIssueの傾向として扱う.

% ========== ========== ========== ========== ========== %
%                       separator                        %
% ========== ========== ========== ========== ========== %

\section{\ref{enum:statistical}}
\label{sec:statistical}

本手順では\ref{enum:aggregate}で得られたIssueをピアソンの\(\chi^2\)検定により,各ラベルの出現割合をウェルチの検定によりそれぞれ比較する.
ここでピアソンの\(\chi^2\)検定は分類されたデータの連関を検定するノンパラメトリック検定であり,ウェルチの検定は2つの平均の差を検定するパラメトリック検定である.
本研究においては\ref{enum:aggregate}で得られたラベルの出現割合について,形式手法を適用して開発されたソフトウェアとそうでないソフトウェアとの間でピアソンの\(\chi^2\)検定を行う.
また各ラベルの出現割合について,形式手法を適用して開発されたソフトウェアとそうでないソフトウェアとの間でウェルチの検定(片側検定)を行う.

ただし検定はotherのラベルについての情報を除いて行った.
これはBERTopicによる分類における外れ値にotherを付与しており,どの分類にも属さないIssueのみがotherのラベルを持つとは限らないためである.

\end{document}

