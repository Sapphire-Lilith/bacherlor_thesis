%!TEX root = ../main.tex
\documentclass[main]{subfiles}

\begin{document}

\chapter{背景}

\section{仕様}

ソフトウェア開発において仕様の品質の向上がソフトウェアの品質の向上に寄与することが報告されている\cite{knauss:2009}.

そのため仕様の品質を向上させる手法が研究されており,自然言語で記述された仕様の曖昧さを低減する手法が提案されている\cite{koerner:2011}.

\section{形式手法}

ソフトウェア開発において仕様の品質を向上させる手法として形式手法が注目されている\cite{aoki:2018}.

形式手法はシステムの仕様を形式的に記述・検証する手法である.
ここで形式的とは数学的に厳密であることを意味し,形式的に記述された仕様は特定の性質を満たすことを容易に検証できる.

KoernerとBrummは自然言語で記述された仕様から曖昧さを低減する手法を提案しているが,形式的に記述された仕様は厳密であるため曖昧さを低減する必要がない.
また形式的に記述された仕様は自然言語で記述された仕様に対して高品質であることが報告されており\cite{fabbrini:2001},自然言語を形式的な言語に変換する手法が提案されている\cite{ilieva:2005}.



\end{document}
