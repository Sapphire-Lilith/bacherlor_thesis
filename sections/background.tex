%!TEX root = ../main.tex
\documentclass[main]{subfiles}

\begin{document}

\chapter{背景}

\section{仕様}
\label{sec:specification}

ソフトウェア開発において仕様の品質の向上がソフトウェアの品質の向上に寄与することが報告されている\cite{knauss:2009}.

そのため仕様の品質を向上させる手法が研究されており,自然言語で記述された仕様の曖昧さを低減する手法が提案されている\cite{koerner:2011}.

\section{形式手法}
\label{sec:formal-method}

形式手法はシステムの仕様を形式的に記述・検証する手法である.
形式的とは数学的に厳密であることを意味し,形式的に記述された仕様は特定の性質を満たすことを容易に検証できる.

\ref{sec:specification}節で述べたKoernerとBrummの研究は自然言語で記述された仕様の曖昧さを低減するものであるが,形式的に記述された仕様は曖昧さを低減する必要がない.
また自然言語で記述された仕様を形式的に記述された仕様に変換する手法が提案されており\cite{ilieva:2005},既存のソフトウェアに対して形式手法を適用できる.

ここで形式的に記述された仕様の品質は自然言語で記述された仕様の品質に対して高いことが報告されており\cite{fabbrini:2001},ソフトウェアの品質を向上させる手法として形式手法が注目されている.% \cite{aoki:2018}

\section{形式手法の普及}

\ref{sec:formal-method}節で述べた利点にもかかわらず形式手法は一般的なソフトウェア開発手法として普及していない.
この要因として学習コストの高さ\cite{kurita:2011}や開発時間の増加\cite{kitamura:2021}が挙げられており,形式手法の学習に関する研究が行われている\cite{ohnishi:2020,araki:2010,araki:2011}.

一方でこれらはいずれも形式手法が抱える複雑さに起因する課題であり,形式手法の適用によるソフトウェア開発への影響についての研究は行われていない.

\end{document}
