%!TEX root = ../main.tex
\documentclass[main]{subfiles}

\begin{document}

\chapter{結言}

本研究ではソフトウェアに対して報告された問題としてGitHubのリポジトリに対して報告されたIssueに着目し,形式手法を適用したソフトウェア開発に特有の傾向を調査した.
その結果,形式手法を適用したソフトウェア開発とそうでないソフトウェア開発において,Issueの傾向に有意な差が認められた.
このことは形式手法を適用したソフトウェア開発に特有の傾向の存在を示唆するものである.

またソフトウェアの動作不良に関する報告について,形式手法を適用して開発されたソフトウェアがそうでないソフトウェアと比較して有意に少ないことが認められた.
このことは形式手法の適用がソフトウェアの動作不良を低減することを示唆するものである.

一方でソフトウェアの機能追加や改善を求める報告について,形式手法を適用して開発されたソフトウェアがそうでないソフトウェアと比較して有意な差が認められなかった.
このことは形式手法の適用がソフトウェアへの機能の追加や改善の要望に与える影響が小さいことを示唆するものである.

この他にバグ,保守に関する報告についても形式手法を適用して開発されたソフトウェアとそうでないソフトウェアとを比較したが,有意な差は認められなかった.
これらはIssueに付随する議論の内容を考慮していないことやサンプル数の少なさ,トピック分析による外れ値に起因している可能性が考えられるため,今後の調査が求められる.

\end{document}
