%!TEX root = ../main.tex
\documentclass[main]{subfiles}

\begin{document}

\chapter{背景}

\section{ソフトウェア仕様}
\label{sec:specification}

ソフトウェア開発において仕様品質の向上がソフトウェア品質の向上に寄与することが報告されている\cite{knauss:2009}.
ソフトウェア仕様が抱える課題として自然言語による記述に起因する曖昧さが指摘されており\cite{kamsties:2005},自然言語で記述された仕様の曖昧さを低減する手法が提案されている\cite{korner:2009,yang:2011}.
一方でYangらの研究では自然言語で記述された仕様から曖昧さを予測する手法に改善の余地が報告されており,自然言語で記述された仕様から曖昧さを完全に取り除く手法には課題が残されている.
そのためソフトウェア仕様から曖昧さを取り除く手法が求められている.

\section{形式手法}
\label{sec:formal-method}

形式手法はシステムの仕様を形式的に記述・検証する手法である.
形式的とは数学的に厳密であることを意味し,形式的に記述された仕様は特定の性質を満たすことを容易に検証できる.
\ref{sec:specification}節で述べたKoernerとBrummの研究\cite{korner:2009}は自然言語で記述された仕様の曖昧さを低減するものであるが,形式的に記述された仕様は曖昧さを低減する必要がない.
また既存のソフトウェア開発に対して形式手法を適用する手法として自然言語で記述された仕様を形式的に記述された仕様に変換する手法が提案されている\cite{ilieva:2005}.
形式的に記述された仕様は自然言語で記述された仕様の品質より品質に優れることが報告されており\cite{fabbrini:2001},ソフトウェア仕様の品質を向上させる手法として形式手法が注目されている.% \cite{aoki:2018}
ソフトウェア仕様の品質の向上はソフトウェア品質の向上に寄与するため,形式手法はソフトウェア工学における重要な題材の1つとなっており,その普及が求められている.

\section{形式手法の普及}

\ref{sec:formal-method}節で述べた利点にもかかわらず形式手法は一般的なソフトウェア開発手法として普及しているとは言えない.
この要因として学習コストの高さが報告されており\cite{kurita:2011},形式手法の学習に関する研究が行われている\cite{ohnishi:2020,araki:2010,araki:2011}.
また開発時間の増加もその要因として報告されている\cite{kitamura:2021}が,これらはいずれも形式手法が抱える複雑さに起因する課題であり,形式手法の適用によるソフトウェア開発への影響についての研究は少ない.
本研究では形式手法を適用したソフトウェア開発に特有の傾向を調査することで形式手法の普及に寄与することを目指す.

\section{GitHub Issues}

GitHub Issues\footnote{https://docs.github.com/en/issues}はGitHub\footnote{https://github.com}リポジトリに提供される機能であり,リポジトリに対して問題を報告するための機能である.
これによりソフトウェア開発者は他の開発者および使用者からバグの報告や改善の要望を受け取ることができる.
ソフトウェアに報告された問題の分類によってソフトウェア開発を評価した先行研究が存在し\cite{hall:2001},Issueとソフトウェア開発との間に多くの関連が報告されている\cite{bissyande:2013}.
本研究ではGitHub Issuesをデータセットとして形式手法を適用したソフトウェア開発に特有の問題を調査する.

\section{BERTopic}

BERTopicは文書分類に用いられるトピックモデル手法の一種である.
トピックモデル手法としてLDA(Latent Dirichlet Allocation)が広く利用されているが,LDAはハイパーパラメータの調整の難しさが指摘されている\cite{panichella:2021}.
一方でBERTopicはモデルの学習後にトピックの数を調整できるため,LDAより容易に最適な結果を得ることができる.
BERTopicはLDAを含む従来のトピックモデル手法より性能に優れることが報告されている\cite{egger:2022}.
本研究ではGitHub Issuesに対してBERTopicを適用し,報告される問題をトピックモデルにより分類することで形式手法の適用によるソフトウェア開発に特有の問題を調査する.

\end{document}
