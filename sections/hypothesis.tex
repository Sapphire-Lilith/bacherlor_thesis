%!TEX root = ../main.tex
\documentclass[main]{subfiles}

\begin{document}

\chapter{仮説}
\label{chap:hypothesis}

本研究では各リポジトリのIssueに対して\ref{sec:labeling}におけるラベルを付与し,そのラベルに基づいてIssueを分類することで形式手法を適用したソフトウェア開発に特有の傾向を調査する.
ここで形式手法を適用したソフトウェア開発に特有の傾向があると仮定し,加えて次の仮説を立てる.

% textlint-disable ja-technical-writing/sentence-length
\begin{enumerate}[label=仮説\arabic*.]
	\item \label{hyp:1} 形式手法を適用して開発されたソフトウェアは,形式手法を適用せず開発されたソフトウェアと比較してerrorの出現割合が低い.
	\item \label{hyp:2} 形式手法を適用して開発されたソフトウェアは,形式手法を適用せず開発されたソフトウェアと比較してbugの出現割合が高い.
	\item \label{hyp:3} 形式手法を適用して開発されたソフトウェアは,形式手法を適用せず開発されたソフトウェアと比較してimprovementの出現割合が低い.
	\item \label{hyp:4} 形式手法を適用して開発されたソフトウェアは,形式手法を適用せず開発されたソフトウェアと比較してmaintenanceの出現割合が高い.
\end{enumerate}
% textlint-enable ja-technical-writing/sentence-length

\ref{hyp:1}はerrorの出現割合についての仮説である.
errorの出現割合が低いことは,形式手法を適用して開発されたソフトウェアに動作不良の報告が少ないことを示唆する.
したがって\ref{hyp:1}は形式手法を適用して開発されたソフトウェアに動作不良が少ないという仮説である.
これは形式手法を適用して開発されたソフトウェアは仕様が形式的に記述・検証されることで,動作不良が少ないと考えられるために立てたものである.

\ref{hyp:2}はbugの出現割合についての仮説である.
bugの出現割合が高いことは,形式手法を適用して開発されたソフトウェアにバグの報告が多いことを示唆する.
したがって\ref{hyp:2}は形式手法を適用して開発されたソフトウェアにバグが多いという仮説である.
これは形式手法を適用して開発されたソフトウェアは仕様が形式的に記述・検証されることで,バグが発見されやすいと考えられるために立てたものである.

\ref{hyp:3}はimprovementの出現割合についての仮説である.
improvementの出現割合が低いことは,形式手法を適用して開発されたソフトウェアに機能の追加や改善の求める報告が少ないことを示唆する.
したがって\ref{hyp:3}は形式手法を適用して開発されたソフトウェアに機能の追加や改善の要望が少ないという仮説である.
これは形式手法を適用して開発されたソフトウェアは仕様が形式的に記述されることで,仕様の変更が煩雑であると考えられるために立てたものである.

\ref{hyp:4}はmaintenanceの出現割合についての仮説である.
maintenanceの出現割合が高いことは,形式手法を適用して開発されたソフトウェアに保守に関する報告が多いことを示唆する.
したがって\ref{hyp:4}は形式手法を適用して開発されたソフトウェアに保守に関する報告が多いという仮説である.
これは形式手法を適用して開発されたソフトウェアは仕様が形式的に検証されることで,保守が煩雑であると考えられるために立てたものである.

\end{document}
