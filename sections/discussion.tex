%!TEX root = ../main.tex
\documentclass[main]{subfiles}

\begin{document}

\chapter{議論}

本研究では形式手法を適用したソフトウェア開発に特有の傾向を,リポジトリにおいて報告されたIssueに着目して調査した.
その結果,形式手法を適用したソフトウェア開発とそうでないソフトウェア開発において,Issueの傾向に有意な差が認められた.
ここでデータセットとして選択したリポジトリにおいては,形式手法を適用して開発されたソフトウェアとそうでないソフトウェアとのIssue数およびStar数に有意な差は認められなかった.
このことは形式手法を適用したソフトウェア開発に特有の傾向が存在することを示唆している.

ここで表\ref{tab:agg_label}において各データセットにおけるラベルの出現割合のうちerrorとbugとに着目して比較する.

errorについては形式手法を適用して開発されたソフトウェアにおける出現割合が低い.
このことは動作不良が少ないか,または動作不良に関するIssueが報告されにくいことを示唆するものである.
またbugについては形式手法を適用して開発されたソフトウェアにおける出現割合が高い.
このことはバグが多いか,またはバグが発見されやすいことを示唆するものである.

ここで本研究におけるIssueの分類は,Issueのtitle要素およびbody要素に着目して行ったため,errorと判断されたIssueの原因がバグであるかどうかは明確に区別できない.
したがって,本来であればbugと判断されるべきIssueがerrorと判断されている可能性がある.
バグが動作不良の一因であると仮定すると,形式手法を適用して開発されたソフトウェアにバグが多いことと,動作不良が少ないこととは矛盾する.
そのため,以上の傾向からは形式手法を適用して開発されたソフトウェアにおいて動作不良に関するIssueが報告されにくいか,またはバグが発見されやすいと考えられる.
これらはいずれも形式手法の適用がバグを低減する効果があることを示唆するものである.

一方でimprovementについては形式手法を適用して開発されたソフトウェアとそうでないソフトウェアとの間の比率がerrorやbugと比較して小さい.
このことは,形式手法の適用がソフトウェアに対して機能の追加や改善の要望に与える影響が少ないことを示唆するものである.
したがって,形式手法の適用によりソフトウェアの改善が加速される可能性は低いと考えられる.

% textlint-disable ja-technical-writing/ja-no-redundant-expression
ただしラベルの出現割合について各データセットの間でウェルチの検定を行った結果,error,bug,improvementの出現割合についてそれぞれ有意水準5\%で有意な差は認めらかったことに注意する.
そのため本章におけるerrorおよびbugについての傾向は主観的なものであり,形式手法を適用して開発されたソフトウェアに特有のものであるかは明確でない.
ここで検定に用いた各データセットにおけるラベルの出現割合を表\ref{tab:discussion_formal},\ref{tab:discussion_common}にそれぞれ示す.
% textlint-enable ja-technical-writing/ja-no-redundant-expression

以上で議論していないラベルについては,一方のデータセットにおいて全く出現していないため,議論の対象とはしなかった.


\begin{table}[p] % require package "float"
	\centering
	\caption{形式手法を適用して開発されたソフトウェアにおけるラベルの出現割合}
	\label{tab:discussion_formal}
	\begin{tabular}{ccccc} % l:left, c:center, r:right
		% "\\" to newlilne, "\hline" to border
		\hline
		ラベル      & seL4(\%) & hacl-star(\%) & CakeML(\%) & CompCert(\%) \\\hline
		error       & 28.20    & 31.53         & 0.00       & 6.52         \\
		bug         & 9.66     & 0.00          & 0.00       & 17.75        \\
		requirement & 0.00     & 0.00          & 21.10      & 0.00         \\
		improvement & 7.83     & 9.46          & 38.02      & 11.59        \\
		question    & 0.00     & 0.00          & 0.00       & 0.00         \\
		maintenance & 3.92     & 18.47         & 0.00       & 26.81        \\
		none        & 50.39    & 40.54         & 40.88      & 37.32        \\\hline
	\end{tabular}
\end{table}

% label,percentage_seL4_seL4,percentage_hacl-star_hacl-star,percentage_CakeML_cakeml,percentage_AbsInt_CompCert
% error,28.198433420365536,31.53153153153153,0.0,6.521739130434782
% bug,9.660574412532636,0.0,0.0,17.753623188405797
% requirement,0.0,0.0,21.0989010989011,0.0
% improvement,7.83289817232376,9.45945945945946,38.02197802197802,11.594202898550725
% question,0.0,0.0,0.0,0.0
% maintenance,3.91644908616188,18.46846846846847,0.0,26.811594202898554
% none,50.391644908616186,40.54054054054054,40.879120879120876,37.31884057971014


\begin{table}[p] % require package "float"
	\centering
	\caption{形式手法を適用せず開発されたソフトウェアにおけるラベルの出現割合}
	\label{tab:discussion_common}
	\begin{tabular}{ccccc} % l:left, c:center, r:right
		% "\\" to newlilne, "\hline" to border
		\hline
		ラベル      & blog\_os(\%) & cryfs(\%) & shc(\%) & hakyll(\%) \\\hline
		error       & 41.30        & 36.80     & 44.62   & 21.65      \\
		bug         & 0.00         & 3.58      & 0.00    & 0.00       \\
		requirement & 0.00         & 0.00      & 0.00    & 0.00       \\
		improvement & 14.67        & 16.23     & 0.00    & 17.11      \\
		question    & 0.00         & 0.00      & 0.00    & 10.31      \\
		maintenance & 0.00         & 0.00      & 0.00    & 0.00       \\
		none        & 35.78        & 43.44     & 55.38   & 50.93      \\\hline
	\end{tabular}
\end{table}

% label,percentage_phil-opp_blog_os,percentage_cryfs_cryfs,percentage_neurobin_shc,percentage_jaspervdj_hakyll
% error,49.55555555555556,36.754176610978526,44.61538461538462,21.649484536082475
% bug,0.0,3.579952267303103,0.0,0.0
% requirement,0.0,0.0,0.0,0.0
% improvement,14.666666666666666,16.2291169451074,0.0,17.11340206185567
% question,0.0,0.0,0.0,10.309278350515465
% maintenance,0.0,0.0,0.0,0.0
% none,35.77777777777777,43.43675417661098,55.38461538461539,50.92783505154639



\end{document}
