%!TEX root = ../main.tex
\documentclass[main]{subfiles}

\begin{document}

\chapter{議論}

本研究では形式手法を適用したソフトウェア開発に特有の傾向を,ソフトウェアのリポジトリに対して報告されたIssueに着目して調査した.
その結果,形式手法を適用されたソフトウェアとそうでないソフトウェアとの間でIssueの傾向に有意な差が認められた.
このことは形式手法を適用したソフトウェア開発に特有の傾向が存在することを示唆している.
また\ref{sec:hypothesis}節で立てた仮説について,\ref{hyp:1}は支持されたが\ref{hyp:2},\ref{hyp:3},\ref{hyp:4}は支持されなかった.

以下でそれぞれの仮説について考察する.

\ref{hyp:1}が支持されたことは,形式手法を適用して開発されたソフトウェアはそうでないソフトウェアより動作不良の報告が少ないことを示唆する.
したがって形式手法の適用がソフトウェアの動作不良を低減すると考えられる.

\ref{hyp:2}が支持されなかったことは,形式手法を適用して開発されたソフトウェアはそうでないソフトウェアよりバグの報告が多いとは言えないことを示唆する.
各データセットにおけるラベルの出現割合に着目すると,双方ともに半数以上のリポジトリにおいてbugの出現割合が0.0\%であった.
一般にソフトウェアがバグを含まないことは考え難いため,bugが付与されるべきIssueに他のラベルが付与されていると考えられる.
ここでbugが付与されるべきIssueにerrorが付与されていることが考えられる.
Issueにおける動作不良の報告はその後の議論によってbugと判断されることがあるが,本研究ではIssueをtitle要素およびbody要素に着目して分類しており,そのような議論を反映していない.
したがって形式手法の適用がソフトウェアにおけるバグの発見を容易にするとは言えないが,Issueに付随する議論を含めた今後の調査が必要であると考えられる.

\ref{hyp:3}が支持されなかったことは,形式手法を適用して開発されたソフトウェアはそうでないソフトウェアより機能の追加や改善の要求が少ないとは言えないことを示唆する.
したがって形式手法の適用がソフトウェアへの機能の追加や改善の要求に与える影響は小さいと考えられる.

% textlint-disable ja-technical-writing/ja-no-redundant-expression
\ref{hyp:4}が支持されなかったことは,形式手法を適用して開発されたソフトウェアはそうでないソフトウェアより保守に関する報告が多いとは言えないことを示唆する.
ここで形式手法を適用せず開発されたソフトウェアにおけるラベルの出現割合は全てのリポジトリにおいて0.0\%であり,形式手法を適用して開発されたソフトウェアに保守に関する報告が多いように思われる.
一方で各データセットにおけるラベルの出現割合についての検定結果はmaintenanceの出現割合について有意な差を示さなかった.
この一因としてデータセットにおけるリポジトリの少なさが考えられる.
したがって形式手法の適用がソフトウェアにおける保守に与える影響については今後の調査が必要であると考えられる.
% textlint-enable ja-technical-writing/ja-no-redundant-expression

\end{document}
