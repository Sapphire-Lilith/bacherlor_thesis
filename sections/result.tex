%!TEX root = ../main.tex
\documentclass[main]{subfiles}

\begin{document}

\chapter{調査結果}

形式手法を適用して開発されたソフトウェアとそうでないソフトウェアにおけるラベルの出現割合をピアソンの\(\chi^2\)乗検定により有意水準5\%で比較した結果,ラベルの出現割合に有意な差が認められた.
ここで\ref{sec:labeling},\ref{sec:aggregate}節の手順で得た各リポジトリにおけるラベルの出現回数を表\ref{tab:formal_label},\ref{tab:common_label}に示す.
また表\ref{tab:formal_label},\ref{tab:common_label}の平均の列を取り出したものを表\ref{tab:agg_label}に示す.
これが\ref{sec:statistical}節で検定により比較した対象である.

ただし検定はotherのラベルについての情報を除いて行った.
これはBERTopicによる分類における外れ値であり,どの分類にも属さないIssueのみがotherのラベルを持つとは限らないためである.

\input{tables/result_1}

また\ref{sec:hypothesis}節で立てた仮説に用いるerror,bug,improvement,maintenanceのラベルについてウェルチの検定(片側検定)により有意水準5\%で出現割合の大小関係をそれぞれ比較した.
その結果を次に示す.

% textlint-disable ja-technical-writing/sentence-length
\begin{enumerate}
	\item errorの出現頻度は形式手法を適用して開発されたソフトウェアが形式手法を適用せず開発されたソフトウェアと比較して有意に低いことが認められた.
	\item bugの出現頻度は形式手法を適用して開発されたソフトウェアが形式手法を適用せず開発されたソフトウェアと比較して有意に高いことが認められなかった.
	\item improvementの出現頻度は形式手法を適用して開発されたソフトウェアが形式手法を適用せず開発されたソフトウェアと比較して有意に低いことが認められなかった.
	\item maintenanceの出現頻度は形式手法を適用して開発されたソフトウェアが形式手法を適用せず開発されたソフトウェアと比較して有意に高いことが認められなかった.
\end{enumerate}
% textlint-enable ja-technical-writing/sentence-length

また有意な差が認められなかった仮説について大小関係を入れ替えて再度ウェルチの検定(片側検定)により有意水準5\%で比較した結果を次に示す.

% textlint-disable ja-technical-writing/sentence-length
\begin{enumerate}
	\item bugの出現頻度は形式手法を適用して開発されたソフトウェアが形式手法を適用せず開発されたソフトウェアと比較して有意に低いことが認められなかった.
	\item improvementの出現頻度は形式手法を適用して開発されたソフトウェアが形式手法を適用せず開発されたソフトウェアと比較して有意に高いことが認められなかった.
	\item maintenanceの出現頻度は形式手法を適用して開発されたソフトウェアが形式手法を適用せず開発されたソフトウェアと比較して有意に低いことが認められなかった.
\end{enumerate}

ここで検定による比較に用いた各データセットにおけるラベルの出現割合を表\ref{tab:discussion_formal},\ref{tab:discussion_common}にそれぞれ示す.

% textlint-disable

\begin{table}[p] % require package "float"
	\centering
	\caption{形式手法を適用して開発されたソフトウェアにおけるラベルの出現割合}
	\label{tab:discussion_formal}
	\begin{tabular}{ccccc} % l:left, c:center, r:right
		% "\\" to newlilne, "\hline" to border
		\hline
		ラベル      & seL4(\%) & hacl-star(\%) & CakeML(\%) & CompCert(\%) \\\hline
		error       & 28.20    & 31.53         & 0.00       & 6.52         \\
		bug         & 9.66     & 0.00          & 0.00       & 17.75        \\
		requirement & 0.00     & 0.00          & 21.10      & 0.00         \\
		improvement & 7.83     & 9.46          & 38.02      & 11.59        \\
		question    & 0.00     & 0.00          & 0.00       & 0.00         \\
		maintenance & 3.92     & 18.47         & 0.00       & 26.81        \\
		none        & 50.39    & 40.54         & 40.88      & 37.32        \\\hline
	\end{tabular}
\end{table}

% label,percentage_seL4_seL4,percentage_hacl-star_hacl-star,percentage_CakeML_cakeml,percentage_AbsInt_CompCert
% error,28.198433420365536,31.53153153153153,0.0,6.521739130434782
% bug,9.660574412532636,0.0,0.0,17.753623188405797
% requirement,0.0,0.0,21.0989010989011,0.0
% improvement,7.83289817232376,9.45945945945946,38.02197802197802,11.594202898550725
% question,0.0,0.0,0.0,0.0
% maintenance,3.91644908616188,18.46846846846847,0.0,26.811594202898554
% none,50.391644908616186,40.54054054054054,40.879120879120876,37.31884057971014


\begin{table}[p] % require package "float"
	\centering
	\caption{形式手法を適用せず開発されたソフトウェアにおけるラベルの出現割合}
	\label{tab:discussion_common}
	\begin{tabular}{ccccc} % l:left, c:center, r:right
		% "\\" to newlilne, "\hline" to border
		\hline
		ラベル      & blog\_os(\%) & cryfs(\%) & shc(\%) & hakyll(\%) \\\hline
		error       & 41.30        & 36.80     & 44.62   & 21.65      \\
		bug         & 0.00         & 3.58      & 0.00    & 0.00       \\
		requirement & 0.00         & 0.00      & 0.00    & 0.00       \\
		improvement & 14.67        & 16.23     & 0.00    & 17.11      \\
		question    & 0.00         & 0.00      & 0.00    & 10.31      \\
		maintenance & 0.00         & 0.00      & 0.00    & 0.00       \\
		none        & 35.78        & 43.44     & 55.38   & 50.93      \\\hline
	\end{tabular}
\end{table}

% label,percentage_phil-opp_blog_os,percentage_cryfs_cryfs,percentage_neurobin_shc,percentage_jaspervdj_hakyll
% error,49.55555555555556,36.754176610978526,44.61538461538462,21.649484536082475
% bug,0.0,3.579952267303103,0.0,0.0
% requirement,0.0,0.0,0.0,0.0
% improvement,14.666666666666666,16.2291169451074,0.0,17.11340206185567
% question,0.0,0.0,0.0,10.309278350515465
% maintenance,0.0,0.0,0.0,0.0
% none,35.77777777777777,43.43675417661098,55.38461538461539,50.92783505154639

% textlint-enable


\end{document}
