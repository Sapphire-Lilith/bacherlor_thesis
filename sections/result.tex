%!TEX root = ../main.tex
\documentclass[main]{subfiles}

\begin{document}

\chapter{調査結果}

形式手法を適用して開発されたソフトウェアとそうでないソフトウェアにおけるラベルの出現割合をピアソンの\(\chi^2\)乗検定により比較した結果,有意水準5\%でラベルの出現割合に有意な差が認められた.
ここで\ref{sec:labeling},\ref{sec:aggregate}節の手順で得た各リポジトリにおけるラベルの出現回数を表\ref{tab:formal_label},\ref{tab:common_label}に示す.
また表\ref{tab:formal_label},\ref{tab:common_label}の平均の列を取り出したものを表\ref{tab:agg_label}に示す.これが\ref{sec:statistical}節で検定により比較した対象である.
ただし検定はotherのラベルについての情報を除いて行った.
これはBERTopicによる分類における外れ値であり,どの分類にも属さないIssueのみがotherのラベルを持つとは限らないためである.

以上の結果から,形式手法を適用して開発されたソフトウェアとそうでないソフトウェアにおいて,Issueの傾向に有意な差が認められたと言える.

% textlint-disable
\begin{table}[p] % require package "float"
	\centering
	\caption{形式手法を適用して開発されたソフトウェアの各リポジトリにおけるラベルの出現回数}
	\label{tab:formal_label}
	\begin{tabular}{cccccc} % l:left, c:center, r:right
		% "\\" to newlilne, "\hline" to border
		\hline
		ラベル      & seL4(回) & hacl-star(回) & CakeML(回) & CompCert(回) & 合計(回) \\\hline
		error       & 108      & 70            & 0          & 18           & 196      \\
		bug         & 37       & 0             & 0          & 49           & 86       \\
		requirement & 0        & 0             & 96         & 0            & 96       \\
		improvement & 30       & 21            & 173        & 32           & 256      \\
		question    & 0        & 0             & 0          & 0            & 0        \\
		maintenance & 15       & 41            & 0          & 74           & 130      \\
		other       & 193      & 90            & 186        & 103          & 572      \\\hline
	\end{tabular}
\end{table}

% label,count_seL4_seL4,count_hacl-star_hacl-star,count_CakeML_cakeml,count_AbsInt_CompCert,count_sum,percentage
% error,108,70,0,18,196,14.670658682634732
% bug,37,0,0,49,86,6.437125748502995
% requirement,0,0,96,0,96,7.185628742514972
% improvement,30,21,173,32,256,19.16167664670659
% question,0,0,0,0,0,0.0
% maintenance,15,41,0,74,130,9.73053892215569
% none,193,90,186,103,572,42.814371257485035

\begin{table}[p] % require package "float"
	\centering
	\caption{形式手法を適用せず開発されたソフトウェアの各リポジトリにおけるラベルの出現回数}
	\label{tab:common_label}
	\begin{tabular}{cccccc} % l:left, c:center, r:right
		% "\\" to newlilne, "\hline" to border
		\hline
		ラベル      & blog\_os(回) & cryfs(回) & shc(回) & hakyll(回) & 合計(回) \\\hline
		error       & 223          & 154       & 58      & 105        & 540      \\
		bug         & 0            & 15        & 0       & 0          & 15       \\
		requirement & 0            & 0         & 0       & 0          & 0        \\
		improvement & 66           & 68        & 0       & 83         & 217      \\
		question    & 0            & 0         & 0       & 50         & 50       \\
		maintenance & 0            & 0         & 0       & 0          & 0        \\
		other       & 161          & 182       & 72      & 247        & 662      \\\hline
	\end{tabular}
\end{table}

% label,count_phil-opp_blog_os,count_cryfs_cryfs,count_neurobin_shc,count_jaspervdj_hakyll,count_sum,percentage
% error,223,154,58,105,540,36.38814016172507
% bug,0,15,0,0,15,1.0107816711590298
% requirement,0,0,0,0,0,0.0
% improvement,66,68,0,83,217,14.622641509433961
% question,0,0,0,50,50,3.369272237196766
% maintenance,0,0,0,0,0,0.0
% none,161,182,72,247,662,44.60916442048518



\begin{table}[p] % require package "float"
	\centering
	\caption{形式手法を適用して開発されたソフトウェアとそうでないソフトウェアにおけるラベルの出現割合}
	\label{tab:agg_label}
	\begin{tabular}{ccc}
		% "\\" to newlilne, "\hline" to border
		\hline
		ラベル      & 形式手法を適用(\%) & 形式手法を適用せず(\%) \\\hline
		error       & 14.67              & 36.39                  \\
		bug         & 6.44               & 1.01                   \\
		requirement & 7.19               & 0.00                   \\
		improvement & 19.16              & 14.62                  \\
		question    & 0.00               & 3.37                   \\
		maintenance & 9.73               & 0.00                   \\
		other       & 42.81              & 44.61                  \\\hline
	\end{tabular}
\end{table}

% textlint-enable


\end{document}
