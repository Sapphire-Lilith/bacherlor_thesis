%!TEX root = ../main.tex
\documentclass[main]{subfiles}

\begin{document}

\begin{abstract}

	ソフトウェア仕様の品質がソフトウェア品質に寄与することが報告されており,自然言語による仕様の曖昧さが課題として指摘されている.
	形式手法はこの課題に対処する手法として注目されているが,形式手法はソフトウェア開発に広く普及しているとは言えない.
	この要因として学習コストの高さや開発時間の増加などの課題が指摘されているが,これらは形式手法の導入の困難さに焦点が当てており,形式手法の適用によるソフトウェア開発への影響の報告は少ない.
	本研究では,GitHubリポジトリから収集したIssueを用いて形式手法を適用したソフトウェア開発に特有の傾向を調査した.
	具体的には,トピックモデルであるBERTopicを用いてIssueから特徴的な単語および文書を抽出し,これらを用いてIssueをラベリングすることでソフトウェア開発の傾向を調査した.
	調査の結果,形式手法を適用したソフトウェア開発とそうでないソフトウェア開発において,Issueの傾向に有意な差が示された.
	これらの結果は,形式手法を適用したソフトウェア開発に特有の問題が存在することを示唆している.

\end{abstract}

\end{document}
