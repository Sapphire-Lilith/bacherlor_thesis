%! TEX root = ../main.tex
\documentclass[main]{subfiles}

\begin{document}

\chapter{緒言}

% title : 形式検証されたソフトウェアに対して報告された問題のトピックモデルによる傾向調査

% textlint-disable sentence-length,no-doubled-joshi

% ソフトウェア開発において,自然言語による仕様は曖昧さを含むことで誤りの発見が困難になる場合があるため,形式手法が注目されている\cite{formal-methods}.

自然言語で記述された仕様は曖昧であるために特定の性質を満たすことの検証が困難であるが,
形式的に記述された仕様は厳密であるために特定の性質を満たすことの検証が容易である.
このような性質から,信頼性の高いソフトウェア開発手法として形式手法が注目されている\cite{aoki:2018}.

% 一方で現在において形式手法は普及しておらず,その適用は検証が不可欠な領域に限定されている.
一方で現在において形式手法は広く普及しているとは言えず,その適用は形式検証が不可欠な領域に限定されている.
この要因として,学習コストの高さや開発時間の増加などの課題が既に報告されている\cite{reid:2020}.
これらの課題に対して,形式手法の普及を促進するための取り組みが行われている\cite{huisman:2020,ohnishi:2020}.

しかし既に報告された課題はいずれも形式手法の導入が困難であることに焦点を当てており,形式手法の適用による問題については報告されていない.
そこで本研究ではソフトウェアに対して報告される問題に焦点を当て,形式検証されたソフトウェアに特有の問題を調査する.

調査にはGitHubリポジトリから取得したIssueをデータセットとして,トピックモデルであるBERTopic\cite{grootendorst:2022}を手法としてそれぞれ用いる.
データセットに対してトピックモデルを適用することでIssueから特徴的な単語を抽出し,形式手法を適用したソフトウェア開発に特有の問題を調査する.

\todo{結果を書く}

% \todo{構成を書く}
本報告の章構成は次の通りである.
第2章で本研究に関わる背景知識について述べる.
第3章で本研究の目的および調査対象とするデータセットの概要について述べる.
第4章で調査方法について述べる.
第5章で調査結果について述べる.
第6章で調査結果の妥当性について議論する.
第7章で結言を述べる.

\end{document}

% 形式手法に普及して欲しいが,なかなか普及しない
% この一因が学習・記述コストの高さだが,開発・保守にも特有の問題があるのでは?
% その問題を解消する手法がわかれば,もっと使われるのでは?
% ということで,調査して,一般的なソフトウェアと比較してみた
% 比較するためにOSSを取ってくるときに,Issue数やStar数が近いことを四分位数を用いて説明する

% 形式手法広まってない
% →特有の原因あるはず
% →学習・導入コストの高さは既知,他の原因は?
% →形式手法に特有の問題を発見したい
% →Issueの傾向にに差異があるのでは?
% →なければしゃーない、あれば保守の改善に役立てられそう
