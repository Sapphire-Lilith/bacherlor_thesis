%! TEX root = ../main.tex
\documentclass[main]{subfiles}

\begin{document}

\chapter{緒言}

% textlint-disable sentence-length,no-doubled-joshi

形式手法はシステムの仕様を形式的に記述・検証する手法である.
ここで形式的とは,数学的に厳密であることを指す.

一般にソフトウェアの仕様は自然言語で記述されるが,自然言語で記述された仕様は曖昧さを含むことがあるため,誤った解釈による実装が発生し得る.
一方で形式的に記述された仕様は,数学的に厳密であるため,誤った解釈による実装が発生しない.

また形式的に記述された仕様は,検証ツールを用いて特定の性質を満たすことを検証できるため,仕様に誤りがないことを保証できる.

このような性質のために,形式手法は信頼性の高いソフトウェアの開発に有用であるとされている.



\end{document}


% 近年、ソフトウェア開発における形式手法の利用が注目されています。
% 形式手法は、数学的なモデルや厳密な仕様に基づいてソフトウェアを設計および検証する手法であり、その適用によってバグの発生を減らし、信頼性の高いソフトウェアの開発を可能にします。
% しかしながら、現在の形式手法は一部の特定領域に限定された使用が主流であり、一般的なソフトウェア開発においてはあまり採用されていません。
% その理由として、形式手法の導入には高い学習コストや開発時間の増加などの課題が挙げられます。

% 特に、オープンソースソフトウェア(OSS)において形式手法を導入する場合、その影響や特有の問題についての理解が不十分です。
% 本研究では、形式手法を導入したOSSとそうでないOSSにおける報告された問題に着目し、その違いを明らかにすることを目指します。
% 具体的には、GitHubリポジトリから取得したIssueデータを用いて、自然言語処理とトピックモデリングの手法を組み合わせて分析を行います。

% この研究の成果は、形式手法の導入に関連する課題を明らかにすることで、形式手法の普及と発展に寄与するものと期待されます。
% 形式手法をより一般的に利用するためには、その特有の問題や課題についての理解が不可欠であり、本研究がその一助となることを目指します。



% 形式手法はシステムの仕様を形式的に記述,検証する手法である.
% ここで形式的とは,数学的に厳密であることを指す.

% 一般に,ソフトウェアの仕様は自然言語によって記述されるが,自然言語によって記述された仕様は曖昧さを含むことがあり,誤った解釈による実装が発生し得る.

% 自然言語によって記述された仕様は,曖昧さを含むことがあり,誤った解釈による実装が発生し得る.
% 一方形式的に記述された仕様は,数学的に厳密であるため,誤った解釈による実装が発生しない.
% 加えて,形式的に記述された仕様は,検証ツールを用いて検証できる.



% ソフトウェア開発において、形式手法は高い信頼性や品質を確保するための有力な手法として注目されています。
% しかし、現実には形式手法の普及が十分に進んでいないという課題が依然として残っています。
% この論文では、形式手法が広まらない原因を探り、解決策を模索することを目的とします。
% 形式手法の学習・導入コストの高さは広く知られていますが、それ以外にも形式手法の普及を妨げる特有の原因が存在すると考えられます。
% 本研究では、これらの特有の原因を明らかにすることを目指します。

% さらに、形式手法には特有の問題が潜んでいる可能性があります。
% これらの問題を特定し、克服することが形式手法の普及につながると考えられます。

% また、異なるプロジェクトやIssueにおける形式手法の適用状況や効果には差異があるかもしれません。
% 本研究では、このような差異を分析し、形式手法の適用範囲や改善点を明らかにします。

% 最終的に、形式手法の普及が進まない理由を理解し、その解決策を提案することは、ソフトウェア開発の品質向上に貢献することが期待されます。



% 形式手法に普及して欲しいが,なかなか普及しない
% この一因が学習・記述コストの高さだが,開発・保守にも特有の問題があるのでは?
% その問題を解消する手法がわかれば,もっと使われるのでは?
% ということで,調査して,一般的なソフトウェアと比較してみた
% 比較するためにOSSを取ってくるときに,Issue数やStar数が近いことを四分位数を用いて説明する

% 形式手法広まってない
% →特有の原因あるはず
% →学習・導入コストの高さは既知,他の原因は?
% →形式手法に特有の問題を発見したい
% →Issueの傾向にに差異があるのでは?
% →なければしゃーない、あれば保守の改善に役立てられそう
