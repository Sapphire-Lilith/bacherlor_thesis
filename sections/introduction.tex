%! TEX root = ../main.tex
\documentclass[main]{subfiles}

\begin{document}

\chapter{緒言}

ソフトウェア開発において仕様品質の向上がソフトウェア品質の向上に寄与することが報告されており\cite{knauss:2009},ソフトウェア仕様が抱える課題として自然言語による記述に起因する曖昧さが指摘されている\cite{kamsties:2005}.
形式手法はシステムの仕様を形式的に記述・検証する手法であり,形式的に記述された仕様は特定の性質を容易に検証できる.
このような性質から,高品質なソフトウェアの開発手法として形式手法が注目されている\cite{aoki:2018}.

一方で現在において形式手法は広く普及しているとは言えず,その適用は形式検証が不可欠な領域に限定されている.
この要因として学習コストの高さや開発時間の増加などの課題が報告されており\cite{reid:2020,kitamura:2021},これらの課題を解消する取り組みが行われている\cite{huisman:2020,ohnishi:2020}.

しかしこれらの課題は形式手法の導入の難しさに焦点を当てており,形式手法の適用によるソフトウェア開発への影響の報告は少ない.
そこで本研究ではソフトウェアに対して報告される問題に焦点を当て,形式手法を適用したソフトウェア開発に特有の傾向を調査した.

調査にはGitHubリポジトリから取得したIssueをデータセットとして,トピックモデルであるBERTopic\cite{grootendorst:2022}を手法として,それぞれ用いた.
データセットに対してトピックモデルを適用することでIssueから特徴的な単語および文書を抽出し,抽出された特徴によってIssueにラベルを付与した.
このラベルによりIssueを分類し,形式手法を適用したソフトウェア開発に特有の傾向を調査した.

本研究では形式手法を適用したソフトウェア開発に特有の傾向が存在することを仮説とした.
またこの仮説の下で,ソフトウェアに対して報告される動作不良,バグ,機能の追加や改善,保守の4項目に関する報告について形式手法を適用して開発されたソフトウェアに特有の傾向が存在すると仮説を立てた.

調査の結果,形式手法を適用したソフトウェア開発とそうでないソフトウェア開発との間において,Issueの傾向に有意な差が認められた.
またソフトウェアの動作不良に関する報告について,形式手法を適用して開発されたソフトウェアがそうでないソフトウェアより有意に少ないことが認められた.
一方でソフトウェアの機能追加や改善を求める報告について,形式手法を適用して開発されたソフトウェアとそうでないソフトウェアとの間に有意な差は認められなかった.
これらは形式手法の適用がソフトウェア開発に特有の影響を与えること,ソフトウェアの動作不良を低減すること,ソフトウェアへの機能の追加や改善の要望に与える影響が小さいことをそれぞれ示唆するものである.

本報告の章構成は次の通りである.
第2章で本研究に関わる背景知識について述べる.
第3章で本研究における調査の目的および対象とするデータセットの概要について述べる.
第4章で調査方法について述べる.
第5章で調査結果について述べる.
第6章で調査結果について議論する.
第7章で本研究が抱える課題と今後の展望について述べる.
第8章で結言を述べる.

\end{document}
