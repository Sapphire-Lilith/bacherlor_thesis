%! TEX root = ../main.tex
\documentclass[main]{subfiles}

\begin{document}

\chapter{緒言}

% title : 形式検証されたソフトウェアに対して報告された問題のトピックモデルによる傾向調査

% textlint-disable sentence-length,no-doubled-joshi

% ソフトウェア開発において,自然言語による仕様は曖昧さを含むことで誤りの発見が困難になる場合があるため,形式手法が注目されている\cite{formal-methods}.

自然言語で記述された仕様は曖昧であるために特定の性質を満たすことの検証が困難であるが,
形式的に記述された仕様は厳密であるために特定の性質を満たすことの検証が容易である.
このような性質から,信頼性の高いソフトウェア開発手法として形式手法が注目されている\cite{aoki:2018}.

一方で現在において形式手法は普及しておらず,その適用は検証が不可欠な領域に限定されている.
この要因として,学習コストの高さや開発時間の増加などの課題がすでに報告されている\cite{reid:2020}.
これらの課題に対して,形式手法の普及を促進するための取り組みが行われている\cite{huisman:2022,ohnishi:2020}.

しかしすでに報告された課題はいずれも形式手法の導入が困難であることに焦点を当てており,形式手法の適用による問題については報告されていない.
そこで本研究ではソフトウェアに対して報告される問題に焦点を当て,形式検証されたソフトウェアに特有の問題を調査する.

調査にはGitHubリポジトリから取得したIssueをデータセットとして,トピックモデルであるBERTopic\cite{bertopic}を手法としてそれぞれ用いる.
データセットに対してトピックモデルを適用することでIssueから特徴的な単語を抽出し,形式検証されたソフトウェアに対して報告された問題の傾向を調査する.

(結果)

本報告の章構成は次の通りである.
第2章で本研究に関わる知識について述べる.
第3章で,本研究の目的および調査対象とするデータセットの概要について述べる.

% TODO
第 章で調査結果およびその妥当性について議論する.
第 章で結言を述べる.

\end{document}


% しかし現在において形式手法を適用したソフトウェアは限られており,一般的なソフトウェア開発においてはあまり採用されていない.
% この原因として,形式手法の導入には高い学習コストや開発時間の増加などの課題が挙げられる.


% 現在において,ソフトウェア開発における形式手法の適用は形式手法を採用する理由を持つアプリケーションに限定されている.
% これは形式手法の適用がソフトウェア開発



% ソフトウェアシステムの開発現場において,自然言語記述による仕様には曖昧な表現が含有されているという理由から内容の矛盾や誤りの指摘が困難になる場合があるため,
% 仕様の記述が厳密で曖昧性を排除できる形式手法が注目されている [1].
% 一方,形式手法の普及は進んでいるが初歩的な技術利用にとどまる場合が多く,普及推進のための技術教育が求められており [2],
% 以前よりその導入コストの高さと並行して人材の養成確保の問題が課題にあげられている [3].
% 文献 [1] によると,モデル検査ツールを使いこなすには数学や論理学の知識が必要になってくるが,そのつど勉強すればよいとあり,
% 文献 [4] によると,形式手法の導入は簡単ではなく “難しい” が,初めからすべてを先に学ぶ必要は



% 現在、主流の商用ソフトウェア開発プロセスは、ソフトウェアの正しい動作の保証を高めることを目的とした多くの種類の正式な手法の範囲を超えています。
% 形式的手法の適用はソフトウェア開発プロセスに多大な影響を与えるため、形式的手法の使用を増やすことは困難であり、費用もかかります。
% これにより、形式的手法の適用は、形式的手法を評価する理由のある組織、つまり形式的正当性証明の追加保証を必要とする少数のアプリケーション
% (通常の安全性が重要なシステムなど) に限定されます。
% 私たちが取り組む問題は、形式的な手法のツールや手法へのアクセスと採用をどのようにして増やすことができるかということです。



% 近年、ソフトウェア開発における形式手法の利用が注目されています。
% 形式手法は、数学的なモデルや厳密な仕様に基づいてソフトウェアを設計および検証する手法であり、その適用によってバグの発生を減らし、信頼性の高いソフトウェアの開発を可能にします。
% しかしながら、現在の形式手法は一部の特定領域に限定された使用が主流であり、一般的なソフトウェア開発においてはあまり採用されていません。
% その理由として、形式手法の導入には高い学習コストや開発時間の増加などの課題が挙げられます。

% 特に、オープンソースソフトウェア(OSS)において形式手法を導入する場合、その影響や特有の問題についての理解が不十分です。
% 本研究では、形式手法を導入したOSSとそうでないOSSにおける報告された問題に着目し、その違いを明らかにすることを目指します。
% 具体的には、GitHubリポジトリから取得したIssueデータを用いて、自然言語処理とトピックモデリングの手法を組み合わせて分析を行います。

% この研究の成果は、形式手法の導入に関連する課題を明らかにすることで、形式手法の普及と発展に寄与するものと期待されます。
% 形式手法をより一般的に利用するためには、その特有の問題や課題についての理解が不可欠であり、本研究がその一助となることを目指します。



% ソフトウェア開発において、形式手法は高い信頼性や品質を確保するための有力な手法として注目されています。
% しかし、現実には形式手法の普及が十分に進んでいないという課題が依然として残っています。
% この論文では、形式手法が広まらない原因を探り、解決策を模索することを目的とします。
% 形式手法の学習・導入コストの高さは広く知られていますが、それ以外にも形式手法の普及を妨げる特有の原因が存在すると考えられます。
% 本研究では、これらの特有の原因を明らかにすることを目指します。

% さらに、形式手法には特有の問題が潜んでいる可能性があります。
% これらの問題を特定し、克服することが形式手法の普及につながると考えられます。

% また、異なるプロジェクトやIssueにおける形式手法の適用状況や効果には差異があるかもしれません。
% 本研究では、このような差異を分析し、形式手法の適用範囲や改善点を明らかにします。

% 最終的に、形式手法の普及が進まない理由を理解し、その解決策を提案することは、ソフトウェア開発の品質向上に貢献することが期待されます。



% 形式手法に普及して欲しいが,なかなか普及しない
% この一因が学習・記述コストの高さだが,開発・保守にも特有の問題があるのでは?
% その問題を解消する手法がわかれば,もっと使われるのでは?
% ということで,調査して,一般的なソフトウェアと比較してみた
% 比較するためにOSSを取ってくるときに,Issue数やStar数が近いことを四分位数を用いて説明する

% 形式手法広まってない
% →特有の原因あるはず
% →学習・導入コストの高さは既知,他の原因は?
% →形式手法に特有の問題を発見したい
% →Issueの傾向にに差異があるのでは?
% →なければしゃーない、あれば保守の改善に役立てられそう
