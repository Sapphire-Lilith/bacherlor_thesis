%!TEX root = ../main.tex
\documentclass[main]{subfiles}

\begin{document}

\chapter{今後の展望}

本研究では形式手法を適用したソフトウェア開発に特有の傾向を,リポジトリにおいて報告されたIssueに着目して調査した.
その結果,形式手法を適用したソフトウェア開発とそうでないソフトウェア開発において,Issueの傾向に有意な差が認められた.
この差は形式手法の適用がソフトウェア開発に与える影響を示唆するものである.
一方で個々のラベルについては有意な差が認められなかった.
これはサンプルとなるリポジトリの少なさが原因であると考えられるため,今後はより多くのリポジトリを対象としての調査が望ましいと考えられる.

この他に本研究が抱える課題として次が挙げられる.

% textlint-disable ja-technical-writing/sentence-length
\begin{enumerate}[label=課題\arabic*.]
	\item \label{enum:oss} OSSと産業用ソフトウェアとの差異を考慮していない.
	\item \label{enum:step} リポジトリの開発段階を考慮していない.
	\item \label{enum:domain} BERTopicがIssueに固有の表現を抽出できていない.
	\item \label{enum:outlier} BERTopicが多くの外れ値を生成している.
\end{enumerate}
% textlint-enable ja-technical-writing/sentence-length

\ref{enum:oss}は本研究で対象をOSS(Open Source Software)に限定しているであるため,産業用ソフトウェアについての調査が行われていないことを指す.
産業用ソフトウェアの開発においては,OSSとは異なる事情が存在する.
例えばOSSにおいては開発者は形式手法を独学するが,産業用ソフトウェアにおいては開発者が形式手法について教育を受ける,などである.
そのためOSSと産業ソフトウェアとでは形式手法の適用による影響が異なると考えられる.

\ref{enum:step}はリポジトリの開発段階を考慮していないことを指す.
本研究で調査対象としたリポジトリはStar数およびIssue数のみを選定基準としたが,ソフトウェアのIssueは開発段階によって傾向が異なることが報告されている\cite{bissyande:2013}.
そのためバージョンや開発期間など,リポジトリの開発段階を基準としてリポジトリを選定することが望ましいと考えられる.
ただしGitHubにおいて埋め込みモデル

\end{document}


% そもそもこれOSSだから産業界とは事情が違うよね
% 産業で使うならしっかり教育してから使うはずなので発生する問題が異なるかも
% それに数が少なすぎて統計的に有意な結果が出るかどうかもわからない
% 開発中のリポジトリだったりもするので,開発段階か保守段階かでも傾向が変わるよね
% ドメイン固有の用語で分類されていることがあるのでrepresentation docsで分類した結果とは異なることがある	
% errorは単なる動作不良の報告,bugは仕様と挙動が異なるという報告
