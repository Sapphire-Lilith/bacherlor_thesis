%!TEX root = ../main.tex
\documentclass[main]{subfiles}

\begin{document}

\chapter{今後の展望}

本研究では形式手法を適用したソフトウェア開発に特有の傾向を,リポジトリに対して報告されたIssueに着目して調査した.
その結果,形式手法を適用したソフトウェア開発とそうでないソフトウェア開発との間でIssueの傾向に有意な差が認められた.
この差は形式手法の適用がソフトウェア開発に与える影響を示唆するものである.
一方で個々のラベルについてはerrorのみに有意な差が認められた.
この差は形式手法の適用がソフトウェアの動作不良を低減することを示唆するものである.

ここで本研究が抱える課題として次が挙げられる.

% textlint-disable ja-technical-writing/sentence-length
\begin{enumerate}[label=課題\arabic*.]
	\item \label{enum:outlier} BERTopicによる分類に多くの外れ値が含まれている.
	\item \label{enum:oss} OSSと産業用ソフトウェアとの差異を考慮していない.
	\item \label{enum:step} リポジトリの開発段階を考慮していない.
\end{enumerate}
% textlint-enable ja-technical-writing/sentence-length

\ref{enum:outlier}はBERTopicによる分類において,多くのIssueが外れ値として分類されなかったことを指す.
本研究ではIssueのtitle要素およびbody要素をトピックモデルにより分類するためにBERTopicを用いたが,その結果として多くのIssueが外れ値として分類されなかった.
本研究ではこの外れ値にotherのラベルを付与したため,一部のIssueについて適切なラベルが付与できていないと考えられる.
BERTopicは外れ値を削減する手法を複数持つため,今後はそれらの手法を用いて外れ値を削減することが望ましいと考えられる.

\ref{enum:oss}は本研究で対象をOSS(Open Source Software)に限定しているために産業用ソフトウェアについての調査が行われていないことを指す.
産業用ソフトウェアの開発にはOSSの開発とは異なる事情が存在する.
例えばOSSにおいては開発者は形式手法を独学するが,産業用ソフトウェアにおいては開発者は形式手法について教育を受ける,などである.
そのためOSSと産業用ソフトウェアとでは形式手法の適用による影響が異なると考えられる.

\ref{enum:step}はリポジトリの開発段階を考慮していないことを指す.
本研究で調査対象としたリポジトリはStar数およびIssue数のみを選定条件としたが,Issueはソフトウェアの開発段階によって傾向が変化すると報告されている\cite{bissyande:2013}.
そのためバージョンや開発期間など,リポジトリの開発段階を条件に加えてリポジトリを選定することが望ましいと考えられる.
ただしGitHubにおいて公開されているリポジトリで形式手法を適用して開発されたものは極めて少なく,この条件を加えてリポジトリを選定することは難しい.

今後は以上の課題を解消することにより,形式手法を適用したソフトウェア開発に特有の傾向をより詳細に調査することが望ましいと考えられる.

\end{document}


% そもそもこれOSSだから産業界とは事情が違うよね
% 産業で使うならしっかり教育してから使うはずなので発生する問題が異なるかも
% それに数が少なすぎて統計的に有意な結果が出るかどうかもわからない
% 開発中のリポジトリだったりもするので,開発段階か保守段階かでも傾向が変わるよね
% ドメイン固有の用語で分類されていることがあるのでrepresentation docsで分類した結果とは異なることがある	
% errorは単なる動作不良の報告,bugは仕様と挙動が異なるという報告
