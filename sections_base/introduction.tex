%! TEX root = ../main.tex
\documentclass[main]{subfiles}

\begin{document}

\chapter{緒言}
本報告書の目的は,修士論文・卒業研究報告書の見本を提供することにある.

% textlint-disable sentence-length,no-doubled-joshi

修士論文や卒業研究報告書の執筆は,修士や学部学修の総仕上げであり,
そのねらいは,専門知識を俯瞰して体系化すること,論理的な文章の記述力を
養うこと,時間制約の下で計画的に物事を進める能力を養うこと,にある.
これらの能力は工学の専門家として社会で活躍する上で分野を問わず必須の
能力である.

% textlint-enable sentence-length,no-doubled-joshi

学生諸君には,次のステップに向けて,報告書や論文の執筆に全精力を傾注して
頂きたい.

修士論文や卒業研究報告書は,いわゆる学術論文に準ずるので,「全ての主張や
評価が根拠に基づく」ことが重要である.
「根拠に基づく」とは,文献,統計,実験データなど事実の存在を証明できる
資料を論拠として,そこから合理的に主張や評価を導くことを意味する.
根拠に基づかない主張や評価は,その真偽を他人が検討できないので,著しく
説得力を欠き,単なる感想文と見なされる.
学生諸君には,常に「その主張の根拠は?」と自問しながら論文・報告書を
執筆して頂きたい.

修士論文や卒業研究報告書の書き方,特に体裁や構成を定める目的は,読み手が
把握しやすい適切な文章を書くことにある.
学術論文の書き方については,先人たちが長い試行錯誤を経てある程度の合意を
形成してきた.
文章の書き方も一種の知的財産であり,それを尊重することは,適切な文章を
各自勝手に工夫する,いわゆる「車輪の再発明」の無駄を避けられる点に意義が
ある.
% textlint-disable ja-no-weak-phrase

慣れないうちは,細部に至る書き方の指定を煩雑に思うかも知れないが,それは
主張や評価に根拠がないか,そもそも執筆に必要な時間が確保されていない場合
が多い.
% textlint-enable ja-no-weak-phrase
学生諸君には,先人の知恵の蓄積である書き方の指定を尊重し,守れるだけの
時間のゆとりをもって執筆して頂きたい.

本報告書の構成を紹介する.
まず,論文の書き方のうち,冊子体裁,用字と用語,数式,図・写真・表,脚注
など内容に依らない「体裁」について説明する.
次に,何をどの順に書くのかという「構成」について説明する.
最後に,体裁や構成を守って論文や報告書の執筆にあたって推奨される行動を
紹介して結言とする.
また,参考文献一覧と付録の見本として,書誌情報の記載方法と修士論文pdf
ファイルの提出要項も示す.

\end{document}
