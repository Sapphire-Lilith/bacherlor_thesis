%! TEX root = ../main.tex
\documentclass[main]{subfiles}

\begin{document}

\begin{abstract}
修士論文や卒業研究報告書の執筆は,修士および学部における学修の総仕上げで
あり,そのねらいは,専門知識の俯瞰と体系化,論理的な文章の記述力養成,
時間制約の下で計画的に物事を進める能力の養成,にある.
本報告書は,修士論文・卒業研究報告書の体裁や構成を説明し,書き方の見本を
示すことを目的とする.
修士論文や卒業研究報告書は,いわゆる学術論文に準ずるので,「全ての主張や
評価が根拠に基づいている」こと が重要である.
そのような文章には,読み手が把握しやすい適切な書き方があるが,それを各自
勝手に工夫する,いわゆる車輪の再発明の無駄を避ける点で,体裁や構成を
定める意義がある.
冊子体裁,用字と用語,数式,図・写真・表,脚注など内容に依らない「体裁」
については,原則として電子情報通信学会和文論文誌Dの投稿規定に準拠する.
情報処理学会の投稿規定に準拠してもよいが,信学会規定との混在は認めない.
論文や報告書の構成は,和文概要,英文概要(修士論文のみ),目次,本文+
図表,付録の順とする.
本文は,第1章緒言から第n章結言,謝辞,参考文献の順に書く.
概要は,背景,目的,方法,結果,結論を簡潔明快に1ページ以内で記述し,
本文中に説明のある特定の記号,数式,図表などを引用せず,図表も含めない.
一方,緒言は単独では読まないので,本文中の図,式,文字,記号等を引用し
てもよく,図表を含めてもよい.
結言は本文の一部なので,研究方法と研究結果のみを要約し,緒言は含めず,
後続の章で示した先行研究や従来手法なども含めない.
結言には,達成したことを主に書き,積み残したこと,すなわち今後の課題は
簡潔に書く.
論旨に直接関係ない数式の誘導,引用文献の詳解,実験の詳細なデータなどは,
本文に書かず付録として本文の後ろにまとめて書く.
ソースコードやデータのダンプリストなど,常識的に通読しないものは本文に
も付録にも含めてはならない.
論文や報告書の執筆にあたっては,十分な執筆時間と睡眠時間の確保,朝昼夕の
規則正しい食事,\TeX, Word等の早期習熟,自己推敲の繰り返し,原稿の
バージョン管理と新旧ファイル両方の保存を推奨する.
\end{abstract}

\end{document}
